\documentclass{article}
\usepackage{tabularx}
\usepackage{booktabs}

\begin{document}

\begin{table}[ht]
\centering
\begin{tabularx}{\textwidth}{X p{0.6\textwidth}}
\toprule
\textbf{文件名} & \textbf{说明} \\
\midrule
biodiversity_420Ma_curve.m & 绘制 420Ma 生物多样性曲线(基于 PBDB 数据库) \\
biodiversity_interp_output.m & 对生物多样性插值,以满足 0.5Ma 的精度,写入新文件:“dataset_biodiv_from_420Ma.xlsx” \\
curve_with_Foster_1.m & 将 Foster 数据按方法一模拟后,结合 PBDB 数据投图 \\
Foster_NC_2017_predict_CO2_420.m & 绘制二氧化碳预测值曲线(基于 Foster 发布的数据) \\
npzd_final_edition_Foster_1.m & 基于 Foster 数据的模拟方法一,并将数据投图(箱型图、一维轴图) \\
npzd_final_edition_Foster_2.m & 基于 Foster 数据的模拟方法二 \\
npzd_final_edition_Hu.m & 基于胡永云数据的模拟 \\
T_CO2_analysis_Hu.m & 基于胡永云数据,拟合温度和二氧化碳关系 \\
dataset_biodiv_from_420Ma.xlsx & 420Ma 以来的生物多样性数据(插值后,由 biodiversity_interp_output.m 生成,精度为 0.5Ma) \\
dataset_CO2_biodiv_from_420Ma.xlsx & 420Ma 以来的二氧化碳与生物多样性数据(手动整理后) \\
Foster_NC-2017_predict_CO2_420Ma.xlsx & Foster 发布的 420Ma 以来大气二氧化碳预测值 \\
initial_LiXiang_dataset.xlsx & 胡永云团队 2022 年发表的显生宙数据集 \\
pbdb_biodiversity_data_from_540Ma.xlsx & 540Ma 以来的生物多样性数据(基于 PBDB 数据库) \\
\bottomrule
\end{tabularx}
\caption{文件列表及其说明}
\label{tab:file_list}
\end{table}

\end{document}